\documentclass[11pt]{article}

% Change "review" to "final" to generate the final (sometimes called camera-ready) version.
% Change to "preprint" to generate a non-anonymous version with page numbers.
\usepackage[]{acl}
\usepackage{times}
\usepackage{latexsym}

% For proper rendering and hyphenation of words containing Latin characters (including in bib files)
\usepackage[T1]{fontenc}
% For Vietnamese characters
% \usepackage[T5]{fontenc}
% See https://www.latex-project.org/help/documentation/encguide.pdf for other character sets

% This assumes your files are encoded as UTF8
\usepackage[utf8]{inputenc}
\usepackage{microtype}
\usepackage{inconsolata}
\usepackage{graphicx}

% If the title and author information does not fit in the area allocated, uncomment the following
%
%\setlength\titlebox{<dim>}
%
% and set <dim> to something 5cm or larger.

\title{Group X Progress Report:\\My Group's Project Name}


\author{First Author, Second Author, Third Author \\
  \texttt{\{macid1,macid2,macid3\}@mcmaster.ca} }

%\author{
%  \textbf{First Author\textsuperscript{1}},
%  \textbf{Second Author\textsuperscript{1,2}},
%  \textbf{Third T. Author\textsuperscript{1}},
%  \textbf{Fourth Author\textsuperscript{1}},
%\\
%  \textbf{Fifth Author\textsuperscript{1,2}},
%  \textbf{Sixth Author\textsuperscript{1}},
%  \textbf{Seventh Author\textsuperscript{1}},
%  \textbf{Eighth Author \textsuperscript{1,2,3,4}},
%\\
%  \textbf{Ninth Author\textsuperscript{1}},
%  \textbf{Tenth Author\textsuperscript{1}},
%  \textbf{Eleventh E. Author\textsuperscript{1,2,3,4,5}},
%  \textbf{Twelfth Author\textsuperscript{1}},
%\\
%  \textbf{Thirteenth Author\textsuperscript{3}},
%  \textbf{Fourteenth F. Author\textsuperscript{2,4}},
%  \textbf{Fifteenth Author\textsuperscript{1}},
%  \textbf{Sixteenth Author\textsuperscript{1}},
%\\
%  \textbf{Seventeenth S. Author\textsuperscript{4,5}},
%  \textbf{Eighteenth Author\textsuperscript{3,4}},
%  \textbf{Nineteenth N. Author\textsuperscript{2,5}},
%  \textbf{Twentieth Author\textsuperscript{1}}
%\\
%\\
%  \textsuperscript{1}Affiliation 1,
%  \textsuperscript{2}Affiliation 2,
%  \textsuperscript{3}Affiliation 3,
%  \textsuperscript{4}Affiliation 4,
%  \textsuperscript{5}Affiliation 5
%\\
%  \small{
%    \textbf{Correspondence:} \href{mailto:email@domain}{email@domain}
%  }
%}

\begin{document}
\maketitle
% \begin{abstract}
% \end{abstract}

\section{Introduction}

As researchers are looking for more ways for computers to read the environment, depth perception is becoming an increasingly important feature. Traditionally, depth perception is done with LIDAR and radar. Lidar and radar-based sensors are accurate at long ranges but are more expensive compared to camera-based setups. Cameras, on the other hand do not have proper depth perception without the use of multiple lenses creating a stereo setup. Even then, stereo setups are less accurate than LIDER or radar. 

To attempt the accuracy of LIDAR setups without needing use LIDAR, we are using deep learning techniques to create a model that can predict distances on an image. The model will be trained on taken images along with corresponding distances measured from more accurate LIDAR sources. This problem is still considered an unsolved research problem since it is extremely difficult for models to learn to estimate depth accurately enough for robotics applications using a single image only. 

If the single image depth perception is solved effectively with a robust solution, it will be impactful in many fields, with the primary use being for autonomous features in vehicles. As these vehicles need a large suite of sensors to properly detect their environment, its possible cut costs by reduce the number of sensors needed for proper autonomous function. Cheaper and more accessible depth perception would also encourage other companies to consider and integrate it into products.


\section{Related Work}

Here, talk about the related work you encountered for your approach. Cite at least 5 references. Refer to item 2. No one has done exactly your task? Write about the most similar thing you can find. This should be around 0.25-0.5 pages.

\section{Dataset}

You should write about your dataset here, following the guidelines regarding item 1. This section may be 0.5-1 pages. Depending on your specific dataset, you may want to include subsections for the preprocessing, annotation, etc.

\section{Features}

Describe any features you used for your model, or how your data was input to your model. Are you doing feature engineering or feature selection? Are you learning embeddings? Is it all part of one neural network? Refer to item 2. This may range from 0.25 pages to 0.5 pages.

\section{Implementation}

Describe your model and implementation here. Refer to item 4. This may take around a page.

\section{Results and Evaluation}

How are you evaluating your model? What results do you have so far? What are your baselines? Refer to item 5. This may take around 0.5 pages.

\section{Feedback and Plans}

%Write about your plans for the remainder of the project. This should include a discussion of the feedback you received from your TA, and how you plan to improve your approach. Reflect on your implementation and areas for improvement. Refer to item 6. This may be around 0.5 pages.

Our group mistaking uploaded the wrong file to avenue. Due to this our first feedback was received during a tutorial when our group proposed to the TA. The TA gave us two pieces of feedback. To look into and base our work off what other work have been done on the subject, and use a Convolutional Neural Network. 

Our group decided that this may not be enough feedback and requested a call with a TA. They said that we should add distribution of distances in the dataset, and a baseline for naive random guess, and maybe a multi layer perceptron (simple model)

Our group knew that there was a lot of previous work done on this subject. We used this to influence how we wanted to approach this problem which was shown in our \href{sec:related-work}{Related Work} section. 

The TAs also mentioned two different methods of implementing our model with both CNNs and MLPs. Our group decide that we wanted to use a CNN specifically ResNet 34 as it was designed to be used for computer vision. 

The last bit of feedback is related to creating a baseline for naive random guess. This is something our group may consider but due to the nature of image depth perception a random guess would be very unlikely to yield proper results. 

Some future plans of our project are to use the entre KITTI Dataset, as we are currently using a subset. To do this we would use the McMaster servers for computation as the entire dataset is bigger than 100gb.  



\section{Template Notes}

You can remove this section or comment it out, as it only contains instructions for how to use this template. You may use subsections in your document as you find appropriate.

\subsection{Tables and figures}

See Table~\ref{citation-guide} for an example of a table and its caption.
See Figure~\ref{fig:experiments} for an example of a figure and its caption.


\begin{figure}[t]
  \includegraphics[width=\columnwidth]{example-image-golden}
  \caption{A figure with a caption that runs for more than one line.
    Example image is usually available through the \texttt{mwe} package
    without even mentioning it in the preamble.}
  \label{fig:experiments}
\end{figure}

\begin{figure*}[t]
  \includegraphics[width=0.48\linewidth]{example-image-a} \hfill
  \includegraphics[width=0.48\linewidth]{example-image-b}
  \caption {A minimal working example to demonstrate how to place
    two images side-by-side.}
\end{figure*}


\subsection{Citations}

\begin{table*}
  \centering
  \begin{tabular}{lll}
    \hline
    \textbf{Output}           & \textbf{natbib command} & \textbf{ACL only command} \\
    \hline
    \citep{Gusfield:97}       & \verb|\citep|           &                           \\
    \citealp{Gusfield:97}     & \verb|\citealp|         &                           \\
    \citet{Gusfield:97}       & \verb|\citet|           &                           \\
    \citeyearpar{Gusfield:97} & \verb|\citeyearpar|     &                           \\
    \citeposs{Gusfield:97}    &                         & \verb|\citeposs|          \\
    \hline
  \end{tabular}
  \caption{\label{citation-guide}
    Citation commands supported by the style file.
  }
\end{table*}

Table~\ref{citation-guide} shows the syntax supported by the style files.
We encourage you to use the natbib styles.
You can use the command \verb|\citet| (cite in text) to get ``author (year)'' citations, like this citation to a paper by \citet{Gusfield:97}.
You can use the command \verb|\citep| (cite in parentheses) to get ``(author, year)'' citations \citep{Gusfield:97}.
You can use the command \verb|\citealp| (alternative cite without parentheses) to get ``author, year'' citations, which is useful for using citations within parentheses (e.g. \citealp{Gusfield:97}).

\subsection{References}

\nocite{Ando2005,andrew2007scalable,rasooli-tetrault-2015}

Many websites where you can find academic papers also allow you to export a bib file for citation or bib formatted entry. Copy this into the \texttt{custom.bib} and you will be able to cite the paper in the \LaTeX{}. You can remove the example entries.

\subsection{Equations}

An example equation is shown below:
\begin{equation}
  \label{eq:example}
  A = \pi r^2
\end{equation}

Labels for equation numbers, sections, subsections, figures and tables
are all defined with the \verb|\label{label}| command and cross references
to them are made with the \verb|\ref{label}| command.
This an example cross-reference to Equation~\ref{eq:example}. You can also write equations inline, like this: $A=\pi r^2$.


% \section*{Limitations}

\section*{Team Contributions}

Write in this section a few sentences describing the contributions of each team member. What did each member work on? Refer to item 7.

% Bibliography entries for the entire Anthology, followed by custom entries
%\bibliography{custom,anthology-overleaf-1,anthology-overleaf-2}

% Custom bibliography entries only
\bibliography{custom}

% \appendix

% \section{Example Appendix}
% \label{sec:appendix}

% This is an appendix.

\end{document}
